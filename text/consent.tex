Options to collect consent are minimal. However, respondents can break off the survey at all times, and this is identified at each point of the survey. When first offered the survey, all respondents are offered the first question of the questionnaire, without an introduction (unless the researcher explicitly makes the first question into an introduction or invitation, though this is not usually done). At this time, respondents can choose to answer the question, choose to see a different question, or choose to skip the survey. A link to information about Google Consumer Surveys (but not the researcher) and Google's Privacy policy are prominently displayed. This setup cannot be changed. Figure~\ref{fig:anatomy} shows an example. 
\begin{figure}
	\includegraphics[width=\textwidth]{Selection_359.png}
	\caption{\label{fig:anatomy}Anatomy of a page}
\end{figure}
The same generic setup repeats for each question. After answering the last question, or skipping the survey, the desired article appears. Thus, at all times, all respondents have a way to skip a shown question and still access the premium content.

\cite{doi:10.1093/pan/mpw016} note that non-conforming content can be embedded into the questionnaire using images, however, this cannot be used for all question types.